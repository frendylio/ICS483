\documentclass[11pt]{article}
\usepackage{scribe}
\usepackage{graphicx}

% Uncomment the appropriate line
%\Scribe{Your name}

\Scribes{Frendy Lio Can}
\LectureDate{October 5, 2020}
\LectureTitle{Homework Assignment \#2 - Theory Problems}

%\usepackage[mathcal]{euscript}


\begin{document}

\MakeScribeTop

%\paragraph{This is a paragraph heading} Paragraph.

%%%%%%%%%%%%%%%%%%%%%%%%%%%%%%%%
% PROBLEM 1
%%%%%%%%%%%%%%%%%%%%%%%%%%%%%%%%
\paragraph{\noindent\textbf{\LARGE{Problem 1}}}

% Start of Explaining

\begin{flushleft}
    We know that each pixel corresponds to $0.01 mm$. This implies that the sphered has 
    a projected radius of $20 pixels = 20 * 0.01 = 0.2 mm$. 

    Therefore $0.2mm*Z = 10mm * 1 m \Rightarrow Z = 50$ meters
\end{flushleft}   

%%%%%%%%%%%%%%%%%%%%%%%%%%%%%%%%
% PROBLEM 1
%%%%%%%%%%%%%%%%%%%%%%%%%%%%%%%%
\paragraph{\noindent\textbf{\LARGE{Problem 2 a)}}}

% Start of Explaining

\begin{equation*}
\begin{split}
    O_{cor}[0,0] & = 0*(1) + 1*(1) + 2*(-1) = -1 \\ 
    O_{cor}[0,1] & = 1*(1) + 2*(1) + 3*(-1) = 0 \\ 
    O_{cor}[0,2] & = 2*(1) + 3*(1) + 0*(-1) = 5 \\ 
    O_{cor}[1,0] & = 0*(1) + 4*(1) + 0*(-1) = 4 \\ 
    O_{cor}[1,1] & = 4*(1) + 0*(1) + 5*(-1) = -1 \\ 
    O_{cor}[1,2] & = 0*(1) + 5*(1) + 0*(-1) = 5 \\ 
    O_{cor}[2,0] & = 0*(1) + 0*(1) + 0*(-1) = 0 \\ 
    O_{cor}[2,1] & = 0*(1) + 0*(1) + 1*(-1) = -1 \\ 
    O_{cor}[2,2] & = 0*(1) + 1*(1) + 0*(-1) = 1 
\end{split}
\end{equation*}
\begin{equation*}
\begin{split}
O_{cor} & =  
\begin{bmatrix} -1 &0 &5 \\ 4 &-1 &5 \\ 0 &-1 &1 \end{bmatrix} 
\end{split}
\end{equation*}

%%%%%%%%%%%%%%%%%%%%%%%%%%%%%%%%
% PROBLEM 1
%%%%%%%%%%%%%%%%%%%%%%%%%%%%%%%%
\paragraph{\noindent\textbf{\LARGE{Problem 2 b)}}}

% Start of Explaining

\begin{equation*}
\begin{split}
    O_{cov}[0,0] & = 0*(-1) + 1*(1) + 2*(1) = 3 \\ 
    O_{cov}[0,1] & = 1*(-1) + 2*(1) + 3*(1) = 4 \\ 
    O_{cov}[0,2] & = 2*(-1) + 3*(1) + 0*(1) = 1 \\ 
    O_{cov}[1,0] & = 0*(-1) + 4*(1) + 0*(1) = 4 \\ 
    O_{cov}[1,1] & = 4*(-1) + 0*(1) + 5*(1) = 1 \\ 
    O_{cov}[1,2] & = 0*(-1) + 5*(1) + 0*(1) = 5 \\ 
    O_{cov}[2,0] & = 0*(-1) + 0*(1) + 0*(1) = 0 \\ 
    O_{cov}[2,1] & = 0*(-1) + 0*(1) + 1*(1) = 1 \\ 
    O_{cov}[2,2] & = 0*(-1) + 1*(1) + 0*(1) = 1 
\end{split}
\end{equation*}
\begin{equation*}
\begin{split}
O_{cov} & =  
\begin{bmatrix} 3 &4 &1 \\ 4 &1 &5 \\ 0 &1 &1 \end{bmatrix} 
\end{split}
\end{equation*}

%%%%%%%%%%%%%%%%%%%%%%%%%%%%%%%%
% PROBLEM 3
%%%%%%%%%%%%%%%%%%%%%%%%%%%%%%%%
\paragraph{\noindent\textbf{\LARGE{Problem 3 a)}}}

\begin{equation*}
\begin{split}
h & =  
\begin{bmatrix} 1  &3 &1 \\ 0 &0 &0 \\ -1 &-3 &-1 \end{bmatrix} 
\end{split}
\end{equation*}

%%%%%%%%%%%%%%%%%%%%%%%%%%%%%%%%
% PROBLEM 3
%%%%%%%%%%%%%%%%%%%%%%%%%%%%%%%%
\paragraph{\noindent\textbf{\LARGE{Problem 3 b)}}}

\begin{flushleft}
Yes, h is separable
\end{flushleft} 
\begin{equation*}
\begin{split}
h_1 & =  
\begin{bmatrix} 1  &0 &-1 \end{bmatrix} \\
h_2 & = 
\begin{bmatrix} 1 \\3 \\1\end{bmatrix}
\end{split}
\end{equation*}

\end{document}
